\chapter{Packet structure}
\thispagestyle{empty}% no page number in chapter title page
\begin{table}[htbp]
\caption{Breakdown of packet structure by byte}
\break
\setlength{\arrayrulewidth}{1mm}
\setlength{\tabcolsep}{12pt}
\renewcommand{\arraystretch}{1.5}
 {\rowcolors{3}{whitesmoke}{silver}
\begin{tabular}{ |p{3cm}|p{1.5cm}|p{7cm}|  }
\hline
\multicolumn{3}{|c|}{Packet Structure} \\
\hline
Content & Bytes & Description \\
\hline
Ethernet Header & 0 - 13 & Header due to packaging. \\
IPv4 Header & 14 - 33 & Header due to packaging. \\
UDP Header & 34 - 41 & Header due to packaging. \\
Frame Identifier & 42 - 45 & Binary variable coded as integer, either \textit{keyframe} or not. \\
Frame Number & 46 - 49 & Used to monitor that order of packages is not scrambled. Coded as integer. \\
Segment Number & 50 - 53 & Can be kept track of to detect any drops, coded as an integer. \\
Packet Size & 54 - 57 & For most packets it equals 1538 bites. Coded as long. \\
Event Occurred & 58 - 61 & Uninteresting for our purposes. Coded as float. \\
Sampling time & 62 - 65 & Same as above. Coded as a float. \\
Dummy Data & 66 onward & Necessary filler to achieve specified length. Filled with zeroes. \\
\hline
\end{tabular}
}
\end{table}